\documentclass[a4paper, 11pt]{report}
\begin{document}


\section{parallel computing}

\subsection{What is parallel computing}
The problem when facing large calculations is that they require a lot of computing power and time. Performing these calculations can be done in different types of computation but the most common ones are serial and parallel computation. Serial computing means, you have one compute unit (e.g. a single core CPU) available to deal with all the calculations that have to be done on a certain set of data. The problem will be broken into multiple smaller subproblems that will be solved by a certain instruction. The single compute unit has to perform the instruction on every subproblem to solve the main problem.\\ Parallel computing on the other hand is the simultaneous use of multiple compute units, or a compute resource, to solve a computational problem. We break apart the main problem in smaller subproblems as we do in serial computing. Each part is further broken down to a series of instructions again.Now, since there are multiple compute units, we can distribute the subproblems among these compute units. Every unit can now perform the instruction on their given subproblems simultaneously.\\
One compute resource can constists of multiple compute units, for example it can include multiple processors or it's just a number of computers connected by a network.

\subsection{Subtitle2}

\bibliographystyle{IEEEtran}
\bibliography{References}
\end{document}
