\documentclass[12pt,a4paper,english,twoside,openright]{tutthesis}
\special{papersize=210mm,297mm}

%
% Define your basic information
%

\author{Beckers Lander, Lakiere Henning}
\title{How to distribute efficiently a computation intensive calculation on an Android device  to external compute units with an Android API?}
\titleB{Otsikko}     % translated title for abstract
\thesistype{Master of Science thesis} % or Bachelor of Science, Laboratory Report... 
\examiner{Nurmi Jari} % without title Prof., Dr., MSc or such

% Put your thesis' main language last




% You can also add your own commands
\newcommand\todo[1]{{\color{red}!!!TODO: #1}} % Remark text in braces appears in red
%\newcommand{\angs}{\textsl{\AA}}              % , e.g. slanted symbol for Ångstöm

% Preparatory content ends here



\pagenumbering{roman} % was: {Roman}
\pagestyle{headings}
\begin{document}



% Special trick so that internal macros (denoted with @ in their name)
% can be used outside the cls file (e.g. \@author)
\makeatletter


%
% Create the title page.
% First the logo. Check its language.
\thispagestyle{empty}
\vspace*{-.5cm}\noindent
\includegraphics[width=8cm]{tty_tut_logo}   % Bilingual logo



% Then lay out the author, title and type to the center of page.
\vspace{6.8cm}
\maketitle
\vspace{7.7cm} % -> 6.7cm if thesis title needs two lines

% Last some additional info to the bottom-right corner
\begin{flushright}  
  \begin{minipage}[c]{6.8cm}
    \begin{spacing}{1.0}
      %\textsf{Tarkastaja: Prof. \@examiner}\\
      %\textsf{Tarkastaja ja aihe hyväksytty}\\ 
      %\textsf{xxxxxxx tiedekuntaneuvoston}\\
      %\textsf{kokouksessa dd.mm.yyyy}\\
      \textsf{Examiner: Prof. \@examiner}\\
      \textsf{Examiner and topic approved by the}\\ 
      \textsf{Faculty Council of the Faculty of}\\
      \textsf{xxxx}\\
      \textsf{on 30th July 2014}\\
    \end{spacing}
  \end{minipage}
\end{flushright}

% Leave the backside of title page empty in twoside mode
\if@twoside
\clearpage
\fi



%
% Use Roman numbering I,II,III... for the first pages (abstract, TOC,
% termlist etc)
\pagenumbering{Roman} 
\setcounter{page}{0} % Start numbering from zero because command 'chapter*' does page break


% Some fields in abstract are automated, namely those with \@ (author,
% title in the main language, thesis type, examiner).
\chapter*{Abstract}

\begin{spacing}{1.0}
         {\bf \textsf{\MakeUppercase{\@author}}}: \@title\\   % use \@titleB when thesis is in Finnish
         \textsf{Tampere University of Technology}\\
         \textsf{\@thesistype, xx pages, x Appendix pages} \\
         \textsf{xxxxxx 201x}\\
         \textsf{Master's Degree Programme in xxx Technology}\\
         \textsf{Major: }\\
         \textsf{Examiner: Prof. \@examiner}\\ % 
         \textsf{Keywords: }\\
\end{spacing}


The abstract is a concise 1-page description of the work: what was the
problem, what was done, and what are the results. Do not include
charts or tables in the abstract.

Put the abstract in the primary language of your thesis first and then
the translation (when that is needed).







\chapter*{Preface}

This document template conforms to Guide to Writing a Thesis at
Tampere University of Technology (2014) and is based on the previous
template. The main purpose is to show how the theses are formatted
using LaTeX (or \LaTeX ~ to be extra fancy) .

The thesis text is written into file \texttt{d\_tyo.tex}, whereas
\texttt{tutthesis.cls} contains the formatting instructions. Both
files include lots of comments (start with \%) that should help in
using LaTeX. TUT specific formatting is done by additional settings on
top of the original \texttt{report.cls} class file. This example needs
few additional files: TUT logo, example figure, example code, as well
as example bibliography and its formatting (\texttt{.bst}) An example
makefile is provided for those preferring command line. You are
encouraged to comment your work and to keep the length of lines
moderate, e.g. <80 characters. In Emacs, you can use \texttt{Alt-Q} to
break long lines in a paragraph and \texttt{Tab} to indent commands
(e.g. inside figure and table environments). Moreover, tex files are
well suited for versioning systems, such as Subversion or Git.  
% \url{http://www.ctan.org/tex-archive/info/lshort/english/lshort.pdf}


Acknowledgements to those who contributed to the thesis are generally
presented in the preface. It is not appropriate to criticize anyone in
the preface, even though the preface will not affect your grade. The
preface must fit on one page. Add the date, after which you have not
made any revisions to the text, at the end of the preface.

~ 
% Tilde ~ makes an non-breakable spce in LaTeX. Here it is used to get
% two consecutive paragraph breaks

Tampere, 11.8.2014

~


On behalf of the working group, Erno Salminen


%
% Term and symbol exaplanations use a special list type
%

\chapter*{List of abbreviations and symbols}
\markboth{}{}                                % no headers
%\chapter*{Lyhenteet ja merkinnät}

% You don't have to align these with whitespaces, but it makes the
% .tex file more readable
\begin{termlist}
\item [CC license] Creative Commons license
\item [LaTeX] 	   Typesetting system for scientific documentation
\item [SI system]  Syst\`eme international d'unités, International System of Units
\item [TUT] 	   Tampere University of Technology
\item [URL]        Uniform Resource Locator
\end{termlist} 

\begin{termlist}
\item [$a$] acceleration
\item [$F$] force
\item [$m$] mass
\end{termlist} 

The abbreviations and symbols used in the thesis are collected into a
list in alphabetical order. In addition, they must be explained upon
first usage in the text.




% The actual text begins here and page numbering changes to 1,2...
% Leave the backside of title empty in twoside mode
\if@twoside
%\newpage
\cleardoublepage
\fi

\pagenumbering{arabic}
\setcounter{page}{1} % Start numbering from zero because command
                     % 'chapter*' does page break
\renewcommand{\chaptername}{} % This disables the prefix 'Chapter' or
                              % 'Luku' in page headers (in 'twoside'
                              % mode)

%%%%%%%%%%%%%%%%%%%%%%%%%%%%%%%%%%%%%%%%%%%%%%%%%%%%%%%%%%%%%%%%%%%%%%%%%%%%%%%%%%%%%%%%%%%%%%%%
%
%%%%%%%%%%%%%%%%%%%%%%%%%%%%%%%%%%%%%%%%%%%%%%%%%%%%%%%%%%%%%%%%%%%%%%%%%%%%%%%%%%%%%%%%%%%%%%%%
%\documentclass[12pt,a4paper,english,twoside,openright]{tutthesis}
%\special{papersize=210mm,297mm}
%\usepackage[finnish, main=english]{babel}
%\usepackage{mathtools}
%\usepackage[utf8]{inputenc}
%\pagenumbering{arabic}
%\setcounter{page}{1} % Start numbering from zero because command
%                     % 'chapter*' does page break
%\renewcommand{\chaptername}{} % This disables the prefix 'Chapter' or
%                              % 'Luku' in page headers (in 'twoside'
%                              % mode)
%
%\begin{document}

\chapter{Introduction}
	


\bibliographystyle{IEEEtranS}
\bibliography{ThesisReferences}
%\end{document}  %\documentclass[12pt,a4paper,english,twoside,openright]{tutthesis}
%\special{papersize=210mm,297mm}
%\usepackage[finnish, main=english]{babel}
%\usepackage{mathtools}
%\usepackage[utf8]{inputenc}
%\pagenumbering{arabic}
%\setcounter{page}{1} % Start numbering from zero because command
%                     % 'chapter*' does page break
%\renewcommand{\chaptername}{} % This disables the prefix 'Chapter' or
%                              % 'Luku' in page headers (in 'twoside'
%                              % mode)
%
%\begin{document}

\chapter{Theoretical background}
	\section{Matrix multiplication}
Before explaining why we chose a matrix multiplication, we will show how a matrix A and B are multiplied in equation \ref{eq:matrixMultiplication}
\begin{equation} \label{eq:matrixMultiplication}
	 \begin{bmatrix}
	1 & 2 \\[0.2em]
	3 & 4
	\end{bmatrix}
	*
	 \begin{bmatrix}
	5 & 6 \\[0.2em]
	7 & 8
	\end{bmatrix}
	=
	 \begin{bmatrix}
	1*5+2*7 & 1*6+2*8 \\[0.2em]
	13*5+4*7 & 3*6+4*8
	\end{bmatrix}
	=
	 \begin{bmatrix}
	19 & 22 \\[0.2em]
	93 & 50
	\end{bmatrix}
\end{equation}
When a problem is hard, people commonly divide the hard problem in multiple smaller/easier problems. Using matrices is one of those easier ways to deal with data, that's why the most common tools in electrical engineering and computer science are rectangular grids of numbers known as matrices. The numbers in a matrix can represent data, and they can also represent mathematical equations. In many time-sensitive engineering applications, multiplying matrices can give quick but good approximations of much more complicated calculations \cite{Hardesty2013}. So the first reason why a matrix multiplication was chosen is the importance of calculations with matrices.\\
The second reason consist of the increasing computations when a matrix grows. The relation between the size of the matrix and the calculation is the following:
\begin{equation} \label{eq:1}
amountOfCalculations = (2*size-1)*size^2
\end{equation}
\begin{equation}
	M =	\begin{bmatrix}
	A & B \\[0.2em]
	.. & ..
	\end{bmatrix}
\end{equation}
\begin{equation}
	N =	\begin{bmatrix}
	F & .. \\[0.2em]
	G & ..
	\end{bmatrix}
\end{equation}
\begin{equation} \label{eq:R1}
	R1 =	\begin{bmatrix}
	A*F+B*G & .. \\[0.2em]
	.. & ..
	\end{bmatrix}
\end{equation}
Deriving equation \ref{eq:1}. First we count the amount of calculations to calculate the first element of the result matrix R1, when processing the matrix multiplication M * N. The amount of calculations is 3, 1 addition of 2 multiplications. When processing the same calculation for a matrix of a bigger size, for example size 5, result matrix \ref{eq:R2} will be calculated. In result matrix R2, 4 additions and 5 multiplications are calculated. Resulting in 4 * 5 calculations in the first element of result matrix R2 with size 5. The number of elements in the matrix is equal to size * size. When you put this all in a formula, we get equation \ref{eq:1}.
\begin{equation} \label{eq:R2}
	R2 =	\begin{bmatrix}
	A*F+B*G+C*H+D*I+E*J & .. & .. & .. & .. \\[0.5em]
	.. & .. & .. & .. & .. \\[0.5em]
	.. & .. & .. & .. & .. \\[0.5em]
	.. & .. & .. & .. & .. \\[0.5em]
	.. & .. & .. & .. & ..
	\end{bmatrix}
\end{equation}





	\section{Parallel computing}
	\section{SoC}
		\subsection{FPGA}
		\subsection{HPS}
		\subsection{Bridges}
			\paragraph{lwHPS2FPGA}
			\paragraph{HPS2FPGA}
			\paragraph{FPGA2HPS}
		\subsection{Quartus}
			\paragraph{Linux RHEL}
			\paragraph{Windows 10}
		\subsection{Bluetooth}
			\paragraph{Master}
			\paragraph{slave}
			\paragraph{Bit error detection}
			\paragraph{Bit error correction}
		\subsection{Web socket}


\bibliographystyle{IEEEtranS}
\bibliography{ThesisReferences}
%\end{document}  
\include{Implementation}


This document template conforms to Guide to Writing a Thesis Tampere
University of technology (TUT) \cite{thesisguide13}. A thesis or a
report typically include the following chapters:
\begin{itemize}
  \setlength{\itemsep}{-10pt} % Put these lines closer to each other
\item[] Title page           % Empty bracket[] remove the bullet
\item[] Abstract
\item[] Preface
\item[] Contents
\item[] List of abbreviations and symbols
\item[] 1. Introduction
\item[] 2. Theoretical background
\item[] 3. Research methodology and materials
\item[] 4. Results and analysis (possibly split into separate chapters)
\item[] 5. Conclusions
\item[] References
\item[] Appendices (if applicable)
\end{itemize}


\section{In-text citations}



Formatting examples of an journal article in bibliography are provided
below, first in the numeric style and then the name-year style.

% Define columns widths to get text wrapped


Section~\ref{sec:3rd} cannot appear alone, but needs some company
(i.e.~\ref{sec:3rd_partner}).



\chapter{Conclusions}
\label{ch:concl}

%
% The bibliography, i.e the list of references (3 options available)
%
\newpage


% Extra for Finnish theses

\renewcommand{\bibname}{Bibliography}     % Bilingual babel puts Finnish ``Kirjallisuttaa'' otherwise. Strange...
%\addcontentsline{toc}{chapter}{Lähteet}  % Include this in TOC
\addcontentsline{toc}{chapter}{\bibname}  % Include this in TOC

\bibliographystyle{IEEEtranS}   % the IEEE's sorted numeric style
\bibliography{thesis_refs}    % Insert {author,title,year...} info of your reference
\markboth{\bibname}{\bibname} % Set page header

\appendix
\pagestyle{headings}

\def\appA{APPENDIX A. Something extra} % Define the name and numbering manually
\chapter*{\appA}                       % Create chapter heading
\markboth{\appA}{\appA}                % Set page header
\addcontentsline{toc}{chapter}{\appA}  % Include this in TOC
% Note that \label does not work with unnumbered chapter

%Appendices are purely optional.  All appendices must be referred to in the body text
%You can append to your thesis, for example, lengthy mathematical
%derivations, an important algorithm in a programming language, input
%and output listings, an extract of a standard relating to your thesis,
%a user manual, empirical knowledge produced while preparing the
%thesis, the results of a survey, lists, pictures, drawings, maps,
%complex charts (conceptual schema, circuit diagrams, structure charts)
%and so on.

\end{document}

